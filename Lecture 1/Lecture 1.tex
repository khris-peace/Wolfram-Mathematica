\documentclass[a4paper,12pt]{article}

%\usepackage{cmap}					% поиск в PDF
\usepackage[T2A]{fontenc}			% кодировка
\usepackage[utf8]{inputenc} 	% кодировка исходного текста
\usepackage{mathtools}			
\usepackage[english,russian]{babel}	% локализация и переносы

\author{Lecture 1}
\title{Wolfram Mathematica}
\date{10/28/2022}


\begin{document} % Конец преамбулы, начало текста.

Functions for construct graphs:

\textbf Plot [function,{variation,a,b}], where [a,b] - is interval of variation

!!!All functions is written with capital letter

Ctr+6 -- write to power of variation; $x^2$

\textbf ContourPlot[function,{variation, a,b}] -- used for implisit functions

ESC a ESC = $\alpha$

Abs ...

Solve[equation] -- Solve equations

Solve[$x^2 == 1$]

Solve[x+y == 1, x] -- solve the equation for x

!!! space = multiplication

Solve[${a+3 b == 7, 5 a - 88 b x == p^2}, (a,b)$]

FullSimplify[expression] -- Simplifying complex expressions

Ctr + '/' -- write fractions; $x/y$

D[function, variation] -- Differentiation

D[$\exp{-x**2}$, {x,n}]

Log@a -- gives the natural logariphm of a (lna)

Log[b,a] -- gives the logarithm to base b $\log_{a}b$

DSolve -- to solve differantial equation

Heads of basic expressions:

1) Symbols

2) String

3) Integer -- целые числа

4) Real -- десятичные дроби

Some commands (Esc + "command" + Esc):

integration: "dintt"

$\infty$: "inf"

partial derivative: "pd"

\end{document}